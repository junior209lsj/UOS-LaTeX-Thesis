%% 여기에 한국어 초록 내용 작성

\indent
\textbf{개요} 본 템플릿은 서울시립대학교 대학원 시행세칙에 의거하여 작성된 비공식 \LaTeX~템플릿이다. 해당 템플릿은 \LaTeX~기본 클래스인 \textit{book}을 기반으로 작성되었으며 컴파일은 \TeX{Live} 2024, pdf\TeX~환경에서 수행되었다.

\textbf{초록의 작성} 학위논문의 초록은 국문, 영문으로 모두 작성해야 한다. 논문을 한글로 작성하는 경우, 국문 초록이 논문의 앞에, 영문 초록이 논문의 뒤에 오고, 논문을 영어로 작성하는 경우, 반대로 영문 초록이 논문의 앞에, 국문 초록이 논문의 위에 온다. 초록 맨 뒤에는 논문의 주제를 잘 나타낼 수 있는 주요어를 작성한다. 한글과 영문 주요어는 각각 \texttt{main.tex}파일의 \texttt{\textbackslash KoKeyword}, \texttt{\textbackslash EnKeyword}에 정의해야 한다. \\

\noindent
\textbf{작성자 정보} \\
이성재 (서울시립대학교 기계정보공학과 실시간시스템연구실 졸업) \\
